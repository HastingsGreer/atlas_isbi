% Template for ISBI paper; to be used with:
%          spconf.sty  - ICASSP/ICIP LaTeX style file, and
%          IEEEbib.bst - IEEE bibliography style file.
% --------------------------------------------------------------------------
\documentclass{article}
\usepackage{spconf,amsmath,graphicx}

\usepackage{enumitem}
\setlist{nosep, leftmargin=14pt}

\usepackage{mwe} % to get dummy images

\title{Author guidelines for ISBI proceedings manuscripts}
%
% Single address.
% ---------------
\name{Author(s) Name(s)\thanks{Some author footnote.}}
\address{Author Affiliation(s)}
%
% For example:
% ------------
%\address{School\\
%	Department\\
%	Address}
%
% Two addresses (uncomment and modify for two-address case).
% ----------------------------------------------------------
%\twoauthors
%  {A. Author-one, B. Author-two\sthanks{Some author footnote.}}
%	{School A-B\\
%	Department A-B\\
%	Address A-B}
%  {C. Author-three, D. Author-four\sthanks{The fourth author performed the work
%	while at ...}}
%	{School C-D\\
%	Department C-D\\
%	Address C-D}
%
% More than two addresses
% -----------------------
% \name{Author Name$^{\star \dagger}$ \qquad Author Name$^{\star}$ \qquad Author Name$^{\dagger}$}
%
% \address{$^{\star}$ Affiliation Number One \\
%     $^{\dagger}$}Affiliation Number Two
%
\begin{document}
\maketitle
\begin{abstract}
	Atlas-building is one of the principle uses of image registration. We want to make a general tool for atlas building based on unigradicon, which does image-to-image registration. We build the atlas. We find that we want to factor out the affine component, which can be defined as "make the approach equivariant to affine transforms" so we get out keymorph. We also try just pre-registering with ANTS to get an affine transform, and an instance optimization hack that turns a deformable registration approach into an affine registration approach. One of these ideas will work better than the others, and we will call the one that works best our innovation, and the others comparison methods.
\end{abstract}
\begin{keywords}
One, two, three, four, five
\end{keywords}
\section{Introduction}
\label{sec:intro}

\section{Methods}

Evaluation:

For all approaches, we measure dice of tibial and femoral cartilage through their atlas. For  evaluating the quality of the affine decomposition, we compute deformation fields with and without an extra affine warp, and verify that they are the same.

ANTs:
ANTs is packaged with an atlas building script. We 











 




In \LaTeX, to start a new column (but not a new page) and help balance the
last-page column lengths, you can use the command ``$\backslash$pagebreak'' as
demonstrated on this page (see the \LaTeX\ source below).

\section{Page numbering}
\label{sec:page}

Please do {\bf not} paginate your paper.  Page numbers, session numbers, and
conference identification will be inserted when the paper is included in the
proceedings.

\section{Illustrations, graphs, and photographs}
\label{sec:illust}

Illustrations must appear within the designated margins.  They may span the two
columns.  If possible, position illustrations at the top of columns, rather
than in the middle or at the bottom.  Caption and number every illustration.
All halftone illustrations must be clear black and white prints.  If you use
color, make sure that the color figures are clear when printed on a black-only
printer.

Since there are many ways, often incompatible, of including images (e.g., with
experimental results) in a \LaTeX\ document, below is an example of how to do
this \cite{Lamp86}.

% Below is an example of how to insert images. Delete the ``\vspace'' line,
% uncomment the preceding line ``\centerline...'' and replace ``imageX.ps''
% with a suitable PostScript file name.
% -------------------------------------------------------------------------
\begin{figure}[htb]

\begin{minipage}[b]{1.0\linewidth}
  \centering
  \centerline{\includegraphics[width=8.5cm]{example-image}}
%  \vspace{2.0cm}
  \centerline{(a) Result 1}\medskip
\end{minipage}
%
\begin{minipage}[b]{.48\linewidth}
  \centering
  \centerline{\includegraphics[width=4.0cm]{example-image}}
%  \vspace{1.5cm}
  \centerline{(b) Results 3}\medskip
\end{minipage}
\hfill
\begin{minipage}[b]{0.48\linewidth}
  \centering
  \centerline{\includegraphics[width=4.0cm]{example-image}}
%  \vspace{1.5cm}
  \centerline{(c) Result 4}\medskip
\end{minipage}
%
\caption{Example of placing a figure with experimental results.}
\label{fig:res}
%
\end{figure}

% To start a new column (but not a new page) and help balance the last-page
% column length use \vfill\pagebreak.
% -------------------------------------------------------------------------
\vfill
\pagebreak


\section{Footnotes}
\label{sec:foot}

Use footnotes sparingly (or not at all!) and place them at the bottom of the
column on the page on which they are referenced. Use Times 9-point type,
single-spaced. To help your readers, avoid using footnotes altogether and
include necessary peripheral observations in the text (within parentheses, if
you prefer, as in this sentence).


\section{Copyright forms}
\label{sec:copyright}

You must include your fully completed, signed IEEE copyright release form when
you submit your paper. We {\bf must} have this form before your paper can be
published in the proceedings.  The copyright form is available as a Word file,
a PDF file, and an HTML file. You can also use the form sent with your author
kit.

\section{Referencing}
\label{sec:ref}

List and number all bibliographical references at the end of the
paper.  The references can be numbered in alphabetic order or in order
of appearance in the document.  When referring to them in the text,
type the corresponding reference number in square brackets as shown at
the end of this sentence \cite{C2}.

\section{Compliance with ethical standards}
\label{sec:ethics}

IEEE-ISBI supports the standard requirements on the use of animal and
human subjects for scientific and biomedical research. For all IEEE
ISBI papers reporting data from studies involving human and/or
animal subjects, formal review and approval, or formal review and
waiver, by an appropriate institutional review board or ethics
committee is required and should be stated in the papers. For those
investigators whose Institutions do not have formal ethics review
committees, the principles  outlined in the Helsinki Declaration of
1975, as revised in 2000, should be followed.

Reporting on compliance with ethical standards is required
(irrespective of whether ethical approval was needed for the study) in
the paper. Authors are responsible for correctness of the statements
provided in the manuscript. Examples of appropriate statements
include:
\begin{itemize}
  \item ``This is a numerical simulation study for which no ethical
    approval was required.'' 
  \item ``This research study was conducted retrospectively using
    human subject data made available in open access by (Source
    information). Ethical approval was not required as confirmed by
    the license attached with the open access data.''
    \item ``This study was performed in line with the principles of
      the Declaration of Helsinki. Approval was granted by the Ethics
      Committee of University B (Date.../No. ...).''
\end{itemize}


\section{Acknowledgments}
\label{sec:acknowledgments}

IEEE-ISBI supports the disclosure of financial support for the project
as well as any financial and personal relationships of the author that
could create even the appearance of bias in the published work. The
authors must disclose any agency or individual that provided financial
support for the work as well as any personal or financial or
employment relationship between any author and the sources of
financial support for the work.

Other types of acknowledgements can also be listed in this section.

Reporting on real or potential conflicts of interests, or the absence
thereof, is required in the paper. Authors are responsible for
correctness of the statements provided in the manuscript. Examples of
appropriate statements include:
\begin{itemize}
  \item ``No funding was received for conducting this study. The
    authors have no relevant financial or non-financial interests to
    disclose.'' 
  \item ``This work was supported by […] (Grant numbers) and
    […]. Author X has served on advisory boards for Company Y.'' 
  \item ``Author X is partially funded by Y. Author Z is a Founder and
    Director for Company C.''
\end{itemize}

% References should be produced using the bibtex program from suitable
% BiBTeX files (here: strings, refs, manuals). The IEEEbib.bst bibliography
% style file from IEEE produces unsorted bibliography list.
% ------------------------------------------------------------------------- 
\bibliographystyle{IEEEbib}
\bibliography{strings,refs}

\end{document}
